% book example for classicthesis.sty
\documentclass[
  % Replace twoside with oneside if you are printing your thesis on a single side
  % of the paper, or for viewing on screen.
  %oneside,
  oneside,
  11pt, a4paper,
  footinclude=true,
  headinclude=true,
  cleardoublepage=empty
]{scrbook}

\usepackage{lipsum}
\usepackage[linedheaders,parts,pdfspacing]{classicthesis}
\usepackage{amsmath}
\usepackage{amsthm}
\usepackage{acronym}
\usepackage{float}
\usepackage{graphicx}
\usepackage{microtype}
\usepackage{hyperref}
\hypersetup{
    colorlinks=true,
    linkcolor=blue,
    filecolor=magenta,      
    urlcolor=cyan,
}
\title{Game Project \\ ``Topdown Game'' Post-Mortem}
\author{Matti Jokitulppo \\ Iiro Surakka \\ Mika Lehtinen}

\begin{document}

\maketitle

\include{FrontBackMatter/contents}

\chapter{Looking back}
\section{What we learned}
\begin{itemize}
\item 3D modeling - Mika didn't have any prior experience with creating 3D assets before the project. Even though he joined the project halfway through, he learned quite a lot during the project.
\item Unreal Engine 4 - During the course, Matti \& Iiro learned a ton about game development with UE4, from gameplay programming to menus, animation, physics and multiplayer.
\item Game project workflow and management - We also gained a lot of new experiences related to managing and scheduling a game project. Our platform of choice was GitHub, where we hosted the whole project and it's related documentation, and we feel this was definitely the right choice.
\end{itemize}

\section{What went wrong}
\begin{itemize}
\item Task priorization - We should've focused first on the core mechanics, to put it simply. Matti \& Iiro just worked on whatever they felt like. This wasn't a big issue all in all, as we got everything done in the end.
\item Milestones - We didn't really keep track of any milestones apart from first playable prototype.
\item Bugs - UE4 turned out to be a lot more unstable than we first though. Lots of random crashes and memory leaks, unexplainable compiling errors and things of that sort. None of the issues were exactly project-destroying, although they hampered the development process quite a bit. It's fair to say however, that there has been a lot of improvement in UE4 since we started back in January.
\end{itemize}
\section{What went right}
\begin{itemize}
\item Game mechanics \& core gameplay - Most importantly of all, we feel we made a game that is actually pretty fun to play
\item Game graphics - For being made by programmers, we think our game is quite pretty. Picking a low-poly art-style and styling it up with different effects and post-processing proved to be a great choice.
\item Choice of version control  - GitHub ended up being a great choice for a small game project such as this. We couldn't  have succeeded without it
\item Choice of documentation - Right from the start, since there were only two primary team members, we decided to keep everything as agile as possible. In practise, that meant keeping documentation short and concise, and letting the game speak for itself, rather than writing large, multi-page design documents and project plans. in retrospect, we feel this was definitely the right choice.
\end{itemize}
\section{Wrapping up}
All in all, we're very pleased with the current state of the project. We got pretty much all of the features implemented we had originally planned (with the notable absence of online multiplayer), and all things considered, the project went over without any major issues.
\chapter{Updated 10 post-mortem points}
Here's our list for assignment 4, with 10 points to keep in mind when working on game projects, along with a note on how we feel about the point in regards to our finished project:
``Topdown Game''
\begin{enumerate}
\item Ambition
    \begin{itemize}
    \item ``Keeping stuff simple doesn't make a bad game, no need to make everything overtly complex with too many gameplay mechanics, enemies, content etc. It's more important to finish the game!''
    
    \medskip
    In our opinion, we handled feature creep pretty efficiently. Since our team has made games before, we could scope the project efficiently right from the start, although naturally, some things were left out.
    \end{itemize}
\item Focus on core mechanics
    \begin{itemize}
    \item ``This one ties into the first point. If you're an indie developer and the master of your own destiny and all that, it's likely you don't have the luxury of spending time on unnecessary features. Therefore, it's important to have a clear, unified vision of what makes the game you're making fun to play.''
    
    \medskip
    This one, we handled pretty well too all in all. We got the basic gameplay nailed rather quickly, and then focused on polishing and improving on it.
    \end{itemize}
\item Don't try to please every gamer
    \begin{itemize}
    \item ``We feel it's important as an indie game developer to get your own message out there in the world, and not compromise on that just to make more money. One of the positive things about going indie is that you don't have to obey publishers or marketing or anybody else but yourself.''
    
    \medskip
    Our game is pretty original in our opinion, merging genres of a bullet hell and top-down action shooter. Neither of the aforementioned genres are all that popular, but we didn't let that stop us from making the game we wanted :).
    \end{itemize}
\item Know your target audience
    \begin{itemize}
    \item ``You should know who you're making your game for, right from the start. As a rule of thumb, if you're making a game you'd really love to play, chances are somebody else would like to play it, too''
    \medskip
    
    We didn't do any market research so to speak, and we just made a game we wanted to make. Nothing wrong with that, in our opinion!
    \end{itemize}
\item Keep the art style simple
    \begin{itemize}
    \item ``Don't go for a too complicated and modern art style. Sure, sweet graphics are fun, but if they deduct valuable development time from the actual game, don't be afraid to scale them back a notch or two. A lot of acclaimed games don't even have good graphics, or graphics at all. (Dwarf Fortress and other roguelikes, Minecraft).''
    
    \medskip
    Like we said earlier, opting for a low-ply art style was definitely the right choice. It allowed us to make clear but simple visuals and assets fast.
    \end{itemize}
\item Getting the atmosphere and mood right
    \begin{itemize}
    \item ``In order for a game to feel "whole" in all of its aspects (graphics, gameplay, sound, story etc.) , it's important that there is a consistent, driven and thought-out idea behind it. If a game has a consistent and fully fleshed out vision behind it, the rest of the pieces fall into place much easier.''
    
    \medskip
    This is a difficult one to evaluate. In our opinion, our game is a pretty cohesive as an experience, mostly thanks to our team being so small, with just two people doing the design and implementation.
    \end{itemize}
\item Beware of unrealistic scope of design and production
    \begin{itemize}
    \item ``Keep the  scope of the game small. You'd rather make a small game that's polished and fun, rather than a larger one that's unfinished all over the place. Be ready to cut features if you don't have enough time to actually finish them.''
    
    \medskip
    Like we said earlier, in our opinion we set the scope of the game correctly. Not much to add to this one.
    \end{itemize}
\item Avoid procrastination, keep tabs on tasks
    \begin{itemize}
    \item ``Make sure every person on team is working on tasks that have concrete, measureable results. Have regular check ups, set milestones and deadlines, imaginary or real, and try to stick to them. Don't let the project fade into apathy as it seems to drag on and on.''
    
    \medskip
    This is the one point where we dropped the ball a little bit. We maybe could've used more milestones and issue tracking, although it didn't really hamper the project in any noticeable way.
    \end{itemize}
\item Play to your strengths
    \begin{itemize}
    \item ``Whether you're an expert programmer or masterful modeller, your team members should focus on what they're good at, and try not to branch out too much. Naturally, in very small teams everyone has to do a bit of everything, but each team member should play to their own strengths.''
    
    \medskip
    There were only two active team members, Matti \& Iiro, with Mika doing some models here and there, so not much need for so-called specializing in such a small team!
    \end{itemize}
\item  Be prepared to fail 
    \begin{itemize}
    \item ``Making games takes a lot of time and resources, and it most likely will not pay out in the end. Make sure you're prepared for the worst, so you won't starve to death if your game doesn't become a smash hit.''
    
    \medskip
    As far as we're concerned, there won't be any money involved in this project, so nothing much to add to this point either.
    \end{itemize}
\end{enumerate}


\chapter{Evaluation}
\section{Team}
We think our team did an excellent job altogether. Even though the workload didn't divide all that evenly between the three members, we'd evaluate the team as a whole as 5.
\section{Self-evaluations}
\subsection{Matti}
``All in all, I think we did a really great job putting this project together. The game is quite good, and has a lot of features and polish.I'd give myself and Iiro a 5, for sure. Iiro lost a bit of steam at the mid-way point due to issues with Unreal Engine, but all in all he did an excellent job. Even though us two did most of the project, including 3d models, I still think Mika was good to have around. It's not really a fair comparison, as me \& Iiro had so much previous experience with game development. Hopefully he learned something about game asset creation. I'd rate him a 2.''
\subsection{Iiro}
``I implemented many of the core features for the game such as weapon, color changing and player movement systems. I addition to that I made a lot of content such as most of the weapons and one map. We managed to make a fun prototype in the given timeframe. I lost a bit of motivation due to frustrating issues with Unreal Engine 4. All in all I would give myself a number 5. I learned a lot about Unreal Engine 4 and its tools.
\medskip
Matti did excellent job creating content for the game. I'd give him a number 5. Mika did few models for the game. His contribution to the game was fairly low. I'd give him a number 2.''

\subsection{Mika}

``Right from the start, this game project turned out to be more of a learning project for me, since I had no previous experience with 3D modelling. During the project, I felt I learned all sorts of things about using Blender, and importing models created with it to Unreal Engine 4. In the end, I managed to create a few usable objects that made it to the final game.

\medskip
Since modelling was totally new to me, I had to unfortunately do a lot of “useless“ work, learning how to use Blender and all. Due to time constraints the amount of my work was somewhat lacking, because behind every model that made it to the final game, there were always a couple of iterations that had to be scratched, due to learning how to do things “the right way”.

\medskip
All in all, I am pretty satisfied with the things I got done, and most of all I learned something about modelling. I’d say I learned the basics of modelling, I am leaving the project with a positive outlook. As for grading, I would currently rate myself a 2. As for Matti and Iiro, they definitely deserve a 5 in my book. They did the vast majority of the work.''

\chapter{Improvements \& suggestions for course}
Here's some pointers for the course for the future:
\begin{itemize}
\item Better organization - Currently, the course was very badly organized. A lot of the time, even the teachers didn't have any idea what to do.

\item Better schedule - The course schedule definitely needed some improvement. Currently, it changed quite drastically a couple of times.

\item Clearer assignments \& tests - The form of the tests weren't decided until very late to the course. The assignment descriptions were also a bit vague, and rather well-hidden in the Optima folders-
\end{itemize}
\chapter{Future of the project}
As of now, the game is a playable prototype. Currently, we're planning on submitting it to Epic's \href{https://www.unrealengine.com/unrealdevgrants}{Unreal Dev Grant program} , and apply for money to further the development, just for fun. Anyways, making it has been a lot of fun, and a learning experience altogether, so it's alright if we won't get any money.
\end{document}